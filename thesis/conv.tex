\documentclass[11pt]{report}

\usepackage[polish]{babel}
\usepackage[latin2]{inputenc}

\title{\textbf{Algorytm ewolucji ró¿nicowej - wp³yw wykorzystania archiwum do mutacji ró¿nicowej na jako¶æ wyników.}}
\author{Andrzej Fiedukowicz}
\date{}

\usepackage{fancyhdr}
\pagestyle{fancy}

\fancyhead{}
\fancyhead[R]{\thepage}
\renewcommand{\chaptermark}[1]{\markboth{\chaptername\ \thechapter.\ #1}{}}
\fancyhead[L]{\leftmark}
\fancyfoot{}
\renewcommand{\headrulewidth}{0.4pt}

\fancyheadoffset[]{0pt}

\fancypagestyle{plain}{
\fancyhf{}
\fancyfoot[C]{\bfseries \thepage}
\renewcommand{\headrulewidth}{0pt}
\renewcommand{\footrulewidth}{0pt}}

\begin{document}

\maketitle
\tableofcontents


%\section{Abstract}
\chapter{Wstêp}
\paragraph{
Problem optymalizacji jest jednym z najczê¶ciej pojawiaj±cych siê w praktycznych zastosowaniach problemów [¬ród³o/Zmiêkczenie].
Jego powszechno¶æ a tak¿e rosn±cy stopieñ skomplikowania konkretnych zadañ stawianych przed systemami informatycznymi rozwi±zuj±cymi ten problem sk³ania do poszukiwania coraz doskonalszych metod jego rozwi±zywania [?].
}
\section{Problematyka / Definicja problemu (motywacje?)} %Tu czy w przegl±dzie literatury??}
\paragraph{
Praca ta, umiejsciowiona zosta³a w kontek¶cie rozwi±zywania problemów optymalizacji gdzie analityczna forma funkcji optymalizowanej nie jest znana lub analityczne wyznaczenie jej ekstremum jest niemo¿liwe. Zak³ada siê bowiem, ¿e zadanie jest zdefiniowane nastêpuj±co [Strojnowski]:
}
\paragraph{
	f: X -> R
	min f(x); x nale¿y do X
Gdzie: 
 	- f - funkcja optymalizowana
	- X - dziedzina funkcji f (zbiór potencjalnych rozwi±zañ)
Dodatkowo zak³ada siê tak¿e mo¿liwo¶æ zbadania warto¶ci funkcji f w dowolnym punkcie ze zbioru X, jednak brak mo¿liwo¶ci bezpo¶redniego zbadania jakichkolwiek innych w³asno¶ci tej¿e funkcji.
}
\paragraph{
Przy tych za³o¿eniach, nale¿y zauwa¿yæ, ¿e je¶li zbiór X jest nieskoñczenie liczny (nawet przeliczalny), nie jest mo¿liwe zweryfikowanie, czy wyznaczony punkt jest rzeczywistym rozwi±zaniem zadanego problemu. Co wiêcej, w wielu przypadkach (brak istnienia hiperp³aszczyzny X, takiej, ¿e w ka¿dym jej punkcie warto¶æ funkcji jest równa poszukiwanemu ekstremum) prawdopodobieñstwo odnalezienia rozwi±zania problemu (niezale¿nie od metody), wynosi zero.
}
\paragraph{
Maj±c to na wzglêdzie by problem mo¿na by³o uznaæ za rozwi±zany, nale¿y go przeformu³owaæ tak, ¿e rozwi±zaniem jest dowolny element x ze zbioru X, jednak jego jako¶æ wyznaczana jest przez warto¶æ funkcji f w tym punkcie. Dla tak sformu³owanego problemu mo¿liwe jest porównywanie ró¿nych metod przegl±dania przestrzeni X pod wzglêdem jako¶ci otrzymanego rozwi±zania.
}
\paragraph{
Warto zwróciæ uwagê, ¿e pomimo i¿ miêdzy ró¿nymi rozwi±zaniami problemu istnieje relacja czê¶ciowego porz±dku generowana przez warto¶æ funkcji f, to jednak przy przyjêtych za³o¿eniach warto¶æ (rozumiana jako liczba rzeczywista) tej funkcji nie niesie ¿adnej dodatkowej informacji. Z tego te¿ powodu, wszelkie badania porównawcze przeprowawdzane na poszczególnych metodykach musz± byæ oparte na tej w³a¶nie relacji porz±dku.
}
\paragraph{
// To chyba mo¿na sobie darowaæ - nie ma znaczenia dla wywodu ?
// Wa¿nym faktem jest to, ¿e je¶li metodê rozwi±zania problemu wraz z jej parametrami uznamy za element zbioru dopuszczalnych metod rozwi±zywania D. To problem wyboru metody optymalizacji staje siê problemem optymalizacyjnym.
}

\section{Motywacje(?)}
\section{Cel}
\section{Proponowana modyfikacja} %tutaj??

\chapter{Przegl±d literatury}
\section{Podejscie heurystyczne do problemu optymalizacji}
\paragraph{
(Czym jest, dlaczego jest stosowane, odnie¶nie do klasycznych algorytmów)
}
\paragraph{
(Algorytm ewolucyjny jako przyk³ad rozwi±zania heurystycznego)
}
\section{Algorytmy ewolucyjne}
\paragraph{
(Czemu heurystyczny?)
}
\paragraph{
(Idea i Schemat dzia³ania)
}
\paragraph{
(Problemy)
}
\paragraph{
(Odmiany i schematy?)
}
\paragraph{
(Parametry)
}
\paragraph{
(Alg. genetyczne - wspomnienie, bez rozwijania w±tku)
}
\paragraph{
(Eksploracja vs. Eksplatacja)
}
\subsection{Ewolucja ró¿nicowa}
\paragraph{
(Ogólny opis)
}
\paragraph{
(Czym siê ró¿ni od klasycznych algorytmów ewolucyjnych)
}
\paragraph{
(Zalety i wady w kontek¶cie problemu optymalizacji)
}
\paragraph{
(Parametry)
}
\paragraph{
(Wyniki w kontek¶cie innych metod)
}
\paragraph{
(Idea punktu ¶rodkowego)
}

\subsubsection{Podej¶cie do archiwum}
\paragraph{(Odniesienia w literaturze) - i gdzie ta literatura :(}
\paragraph{(Dotychczasowe wnioski)}

\section{Metodyka badañ (opisywaæ tak szeroko?)}
\subsection{Modelowanie statystyczne}
\paragraph{
(Czemu s³u¿y, dlaczego potrzebne do tych badañ)
}
\subsection{Testy statystyczne}
\paragraph{
(Ogólna koncepcja, p-value)
}
\subsection{Testy parametryczne}
\paragraph{
(Co to jest i kiedy siê stosuje)
}
\subsubsection{Test T-studenta}
\paragraph{
(Jak dzia³a, ZA£O¯ENIA)
}
\paragraph{
(Wady zalety).
}

\subsection{Testy nieparametryczne}
\paragraph{
(Co to jest i kiedy siê stosuje)
}
\subsubsection{Test wilcoxona}
\paragraph{
(Opis, dzia³anie, ZA£O¯ENIA)
}
\paragraph{
(Wady zalety)
}
\section{Benchmarki blackboxowe}
\paragraph{
(Czemu s³u¿±)
}
\paragraph{
(Dlaczego s± dobrym odniesieniem)
}

\subsection{CEC}
\paragraph{
(Opis benchmarka, funkcje testowe)
}
\paragraph{
(Implementacja w R)
}

%%%%%%%%%%%%%%%%%%% CHAPTER III

\chapter{Badania}
\section{Cel}
\section{Plan eksperymentów}
\section{Dobór parametrów}
\section{Zbierane rezulataty}
\subsection{¦rodek ciê¿ko¶ci populacji (DE/mid)}
\section{Sposób analizy}

\section{Wyniki (w za³±czniku?)}
\section{Analiza wyników (wnioski)}

\chapter{Podsumowanie}


\bibliographystyle{plplain}
\nocite{*}
\bibliography{bibliography}

\end{document}
    <end>$</end>
        </context>

        <context id="escape" style-ref="special-char">
            <match>\\.</match>
        </context>

        <!-- usual quoted string, ends at line end, \ is an escape char -->
        <context id="string" style-ref="string" end-at-line-end="true" class="string" class-disabled="no-spell-check">
            <start>"</start>
            <end>"</end>
            <include>
                <context ref="escape"/>
                <context ref="line-continue"/>
            </include>
        </context>

        <!-- same thing but with single quote marks -->
        <context id="single-quoted-string" style-ref="string" end-at-line-end="true" class="string" class-disabled="no-spell-check">
            <start>'</start>
            <end>'</end>
            <include>
                <context ref="escape"/>
                <context ref="line-continue"/>
            </include>
        </context>

        <!-- Dummy context, needed to load the style mappings when parsing v1 files -->
        <context id="def"/>

    </definitions>
</language>
